\documentclass[12pt]{article}
\usepackage[utf8]{inputenc}
\usepackage{amsmath}
\usepackage{subcaption}
\usepackage{float}
\usepackage{graphicx}
\usepackage{epstopdf}
\usepackage{color}
\newcommand{\de}[1]{$\langle${{\color{cyan}\slshape{\bfseries DE:} #1}$\rangle$}}
\newcommand{\rz}[1]{$\langle${{\color{magenta}\slshape{\bfseries RZ:} #1}$\rangle$}}

\setlength{\parindent}{4em}
\setlength{\parskip}{1em}
\renewcommand{\baselinestretch}{1.5}

\title{Intensional infect proportion of susceptible}

\begin{document}
\maketitle

\de{There is some rate at which susceptibles are intentionally infected, say $r$, so there is a term $-rS$ in the dimensional SIR equations.  We are expressing the SIR model in dimenionless form, so let $\eta=r/(\gamma+\mu)$ and then we get the following.}

Here is the system we are investigating:
\begin{align}
\frac{\mathrm{d}S}{\mathrm{d}\tau}&=\epsilon -\mathcal{R}_0 SI-\eta S-\epsilon S\\
\frac{\mathrm{d}I}{\mathrm{d}\tau}&=\eta S+\mathcal{R}_0 SI-I\\
\frac{\mathrm{d}R}{\mathrm{d}\tau}&=(1-\epsilon)I-\epsilon R
\end{align}

Here $\gamma$ is mean infectious period, $\mu$ is birth/death rate, $r$ is rate of intensional infection. $\epsilon=\frac{\mu}{\gamma+\mu}$, $\mathcal{R}_0$ is the basic reproduction number.

Since last time we discussed that, it is not very meaningful to divide $I$ into $I_T$ and $I_N$. Thus, I just used I this time to investigate the system's equilibrium, stability and other properties.
\section{EE}
Endemic equilibrium is the following:

\begin{align}
I &= \frac{-(\eta+\epsilon-\epsilon\mathcal{R}_0)+\sqrt{(\eta+\epsilon-\epsilon\mathcal{R}_0)^2+4\mathcal{R}_0\epsilon \eta}}{2\mathcal{R}_0}\\
S &= \frac{1}{\mathcal{R}_0}-\frac{2\eta}{\mathcal{R}_0(-(\eta+\epsilon-\epsilon\mathcal{R}_0)+\sqrt{(\eta+\epsilon-\epsilon\mathcal{R}_0)^2+4\mathcal{R}_0\epsilon \eta}+2\eta)}
\end{align}

Jacobian is the following.
\begin{equation}
\mathcal{J} =
\begin{bmatrix}
    \ -\mathcal{R}_0 I-\eta-\epsilon       & -\mathcal{R}_0 S \\
    \ \eta+\mathcal{R}_0 I       & \mathcal{R}_0 S-1 \\
\end{bmatrix}
\end{equation}

Again, for simplicity. Let $G=-(\eta+\epsilon-\epsilon\mathcal{R}_0)+\sqrt{(\eta+\epsilon-\epsilon\mathcal{R}_0)^2+4\mathcal{R}_0\epsilon \eta}$.

So Jacobian at E.E. is:

\begin{equation}
\mathcal{J} =
\begin{bmatrix}
    \ \frac{G}{2}-\eta-\epsilon       & -1+\frac{2L}{G+2\eta} \\
    \ \eta+\frac{G}{2}       & -\frac{2\eta}{G+2\eta} \\
\end{bmatrix}
\end{equation}

\section{DFE}
Here is the analysis on Disease Free Equilibrium (DFE).

Certainly at DFE, $S=1$ and $I=0$. By using the same Jacobian, we get the following:

\begin{equation}
\mathcal{J} =
\begin{bmatrix}
    \ -\eta-\epsilon       & -\mathcal{R}_0 \\
    \ \eta       & \mathcal{R}_0 -1 \\
\end{bmatrix}
\end{equation}

The corresponding eigenvalues are:
\begin{align}
\lambda_1 &=\frac{-(\eta-\mathcal{R}_0+\epsilon+1)+ \sqrt{(\eta-\mathcal{R}_0+\epsilon+1)^2-4(\eta+\epsilon(1-\mathcal{R}_0))}}{2}\\
\lambda_2 &=\frac{-(\eta-\mathcal{R}_0+\epsilon+1)- \sqrt{(\eta-\mathcal{R}_0+\epsilon+1)^2-4(\eta+\epsilon(1-\mathcal{R}_0))}}{2}
\end{align}

Now we want to analyze the with specific values of each parameter.

A reasonable choice would be using smallpox, considering its background of intensional infection in human history.

The following values are used: $\mu=\frac{1}{50*365}$, $\gamma=\frac{1}{22}$, $\mathcal{R}_0=4.5$. Also, $\epsilon=0.0012$

As for value of $\eta$, it is reasonable to assume the average number of days before an individual to be intensionally infected is between 30-60 days. Thus our range of $\eta$ could vary from 0.36623 - 0.73245. Thus, we can conclude that $(\eta+\epsilon(1-\mathcal{R}_0))>0$. As a result, we can also claim that the real part of eigenvalues are always negative. Thus, the DFE is stable. 

Now we are interested in the discriminant of eigenvalue.
\begin{align}
\Delta &=(\eta-\mathcal{R}_0+\epsilon+1)^2-4(\eta+\epsilon(1-\mathcal{R}_0))=(\eta-3.5)^2-4(\eta-0.0042)\\
\Delta &=\eta^2-11\eta+12.2668
\end{align}

If we try to find the $\eta$-intercept of (12), we get:
\begin{align}
\eta_1 &= 1.2593\\
\eta_2 &= 9.7407
\end{align}
The above calculation, we can conclude that with a slower rate of intensional infection ($\eta<1.2593$), DFE is stable. In my opinion, it is unlikely for $\eta>9.7407$ since this means the average time before being intensional infected is about 2 day. Thus, it is meaningless to discuss any $\eta$ values greater than that.

At $\eta\approx 1.2593$, it should be point of where the dynamic of infected individuals start to have damped oscillation. This correspond to an average time before being intensional infected of 17.45 days.

\end{document}