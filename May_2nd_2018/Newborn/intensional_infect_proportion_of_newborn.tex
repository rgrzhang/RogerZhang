\documentclass[12pt]{article}
\usepackage[utf8]{inputenc}
\usepackage{amsmath}

\title{Intensional infect proportion of newborn, finding eigenvalues}

\begin{document}
\maketitle
\begin{align}
\frac{\mathrm{d}S}{d\tau}&=\epsilon(1-p)- \mathcal{R}_0  SI-\epsilon S \\
\frac{\mathrm{d}I}{d\tau}&=\mathcal{R}_0 SI+\epsilon p-I
\end{align}

Here $\gamma$ is mean infectious period, $\mu$ is birth/death rate, $p$ is probability of intensional infection on newborns. $\epsilon=\frac{\mu}{\gamma+\mu}$,$\mathcal{R}_0$ is the basic reproduction number.
\section{EE}
So to find the E.E. Letting both equal to 0, solve get:
\begin{align}
I &= \frac{\epsilon(\mathcal{R}_0 -1)+ \epsilon \sqrt{(\mathcal{R}_0-1)^2+4\mathcal{R}_0 p}}{2\mathcal{R}_0}\\
S &=\frac{1}{\mathcal{R}_0}-\frac{2p}{(\mathcal{R}_0 -1)+ \sqrt{(\mathcal{R}_0-1)^2+4\mathcal{R}_0 p}}\\
\mathcal{R}_0 I &= \frac{\epsilon(\mathcal{R}_0 -1)+ \epsilon \sqrt{(\mathcal{R}_0-1)^2+4\mathcal{R}_0 p}}{2}\\
\mathcal{R}_0 S &= 1-\frac{2p \mathcal{R}_0}{(\mathcal{R}_0 -1)+ \sqrt{(\mathcal{R}_0-1)^2+4\mathcal{R}_0 p}}
\end{align}

Jacobian is the following.
\begin{equation}
\mathcal{J} =
\begin{bmatrix}
    \ -\mathcal{R}_0 I-\epsilon       & -\mathcal{R}_0 S \\
    \ \mathcal{R}_0 I       & \mathcal{R}_0 S-1 \\
\end{bmatrix}
\end{equation}

Now for simplicity, let $(\mathcal{R}_0 -1)+ \sqrt{(\mathcal{R}_0-1)^2+4\mathcal{R}_0 p}$ = $K$

So we get:
\begin{align}
\mathcal{R}_0 I &= \frac{\epsilon K}{2}\\
\mathcal{R}_0 S &= 1-\frac{2p \mathcal{R}_0}{K}
\end{align}

\begin{equation}
\mathcal{J} =
\begin{bmatrix}
    \ -\frac{\epsilon K}{2}-\epsilon       & -1+\frac{2p \mathcal{R}_0}{K} \\
    \ \frac{\epsilon K}{2}       & -\frac{2p \mathcal{R}_0}{K} \\
\end{bmatrix}
\end{equation}

The eigenvalues of this Jacobian is:

\begin{equation}
\lambda = \frac{-(\epsilon K^2+2\epsilon K +4p\mathcal{R}_0) \pm \sqrt{(\epsilon K^2+2\epsilon K +4p\mathcal{R}_0)^2-4(2\epsilon K^3+8\epsilon Kp\mathcal{R}_0)}}{4K}
\end{equation}

Reminder: $K=(\mathcal{R}_0 -1)+ \sqrt{(\mathcal{R}_0-1)^2+4\mathcal{R}_0 p}$

\section{DFE}
Certainly at DFE, $S=1$ and $I=0$. By using the same Jacobian, we get the following:
\begin{equation}
\mathcal{J} =
\begin{bmatrix}
    \ -\epsilon       & -\mathcal{R}_0 \\
    \ 0       & \mathcal{R}_0 -1 \\
\end{bmatrix}
\end{equation}

So eigenvalues are just the entries on the diagonal:
\begin{align}
\lambda _1 &= -\epsilon\\
\lambda _2 &= \mathcal{R}_0 -1
\end{align}
This means, DFE is stable iff $\mathcal{R}_0<1$

\end{document}
