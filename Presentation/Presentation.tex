\documentclass{beamer}
\usetheme{Boadilla}
%%\setlength{\parindent}{4em}
%%\setlength{\parskip}{1em}
%%\renewcommand{\baselinestretch}{1.5}
\newcommand\dbyd[2]{\frac{\mathrm d{#1}}{\mathrm d{#2}}}
\newcommand\dsided[2]{{\mathrm d{#1}}/{\mathrm d{#2}}}
\newcommand{\R}{\mathcal{R}}
\newcommand{\pmV}{p_{V}}
\newcommand{\pmI}{p_{I}}
\title{Intentional infection as a method of population level disease control}
\subtitle{Newborn infection}
\author{Roger Zhang}
\date{June 18th 2018}
\institute{McMaster University Department of Mathematics}

\begin{document}
\begin{frame}
\titlepage
\end{frame}
\begin{frame}
\frametitle{Motivation}
\begin{itemize}
\item What are the historical examples?
\pause
\begin{itemize}
\item Variolation of smallpox
\item Pox party
\end{itemize}
\pause
\item This method is out of date, banned in some places, why do we care?
\pause
\begin{itemize}
\item In history, the mechanisms and benefits of intentional infection  on a population level was not quite understood.
\item New application to immunology, i.e. Transmissible vaccine.
\end{itemize}
\end{itemize}
\end{frame}
%%%%%%%%%%%%%%%%%%%%%%%%%%%%%%%%%%%%%%%%%%%%%%%%%%%%%%%%%%%%%%%
\begin{frame}
\frametitle{Method}
We start our analysis by modifying standard SIR model:
\pause
\begin{equation}\label{1}
\begin{split}
\dbyd{S}{t}&=\mu- \beta SI-\mu S \,,\\
\dbyd{I}{t}&=\beta SI-\gamma I -\mu I\,,\\
\dbyd{R}{t}&=\gamma I-\mu R\,.
\end{split}
\end{equation}

Where $S$, $I$ and $R$ represent susceptible, infected and recovered.

$\mu$ is birth/death rate, $\beta$ is the transmission rate, $\gamma$ is recovery rate. 
\end{frame}
%%%%%%%%%%%%%%%%%%%%%%%%%%%%%%%%%%%%%%%%%%%%%%%%%%%%%%%%%%%%%%%%%%%%%
\begin{frame}
\frametitle{Method}
The following assumptions are made to simplify the model to start with:
\begin{itemize}\itemsep10pt
\item There is no difference between intentionally infected and normally infected individuals.
\item There is no disease induced mortality.
\item Birth and natural death rate are the same, so the total population remains constant.
\item The latent period is short enough to be ignored.
\item All susceptible individuals are equally likely to be infected, and all infected individuals are equally infectious.
\end{itemize}
\end{frame}
%%%%%%%%%%%%%%%%%%%%%%%%%%%%%%%%%%%%%%%%%%%%%%%%%%%%%%%%%%%%%%%%%%
\begin{frame}
\frametitle{Method}
With intentional infection on newborn, we have,
\begin{equation}\label{2}
\begin{split}
\dbyd{S}{t}&=\mu(1-p)- \beta SI-\mu S \,,\\
\dbyd{I}{t}&=\beta SI+\mu p-\gamma I -\mu I\,,\\
\dbyd{R}{t}&=\gamma I-\mu R\,.
\end{split}
\end{equation}

In addition to the parameters above,

$p$ is the proportion of newborn intentionally infected.
\end{frame}
%%%%%%%%%%%%%%%%%%%%%%%%%%%%%%%%%%%%%%%%%%%%%%%%%%%%%%%%%%%%%%%%%%%%%%
\begin{frame}
We non-dimensionalize the above system by scaling time, by
\begin{equation}
\tau=(\gamma+\mu)t \,,
\end{equation}

which yields

\begin{subequations}\label{3}
\begin{align}
\dbyd{S}{\tau}&=\epsilon(1-p)- \R_0  SI-\epsilon S \,,\\
\dbyd{I}{\tau}&=\R_0 SI+\epsilon p-I \,,
\end{align}
\end{subequations}

where $\epsilon=\frac{\mu}{\gamma+\mu}$, $\R_0=\frac{\beta}{\gamma+\mu}$.
\end{frame}
%%%%%%%%%%%%%%%%%%%%%%%%%%%%%%%%%%%%%%%%%%%%%%%%%%%%%%%%%%%%%%%%%%%%%%
\begin{frame}
\frametitle{Equilibria}
Solving equations above to find equilibria, we obtain,
\begin{subequations}
\begin{align}
\hat{S} &=\frac{1}{\R_0}-\frac{2p}{(\R_0 -1)+ \sqrt{(\R_0-1)^2+4\R_0 p}}\,, \label{Shat1}\\
\hat{I} &= \frac{\epsilon(\R_0 -1)+ \epsilon \sqrt{(\R_0-1)^2+4\R_0
    p}}{2\R_0}\,.\label{Ihat1}
\end{align}
\end{subequations}

Notice, $\hat{I}\neq 0$ for all $p$ between 0 and 1. Meaning there is always infected cases in the population. Therefore, the equilibrium is an Endemic Equilibrium (EE).
\end{frame}
%%%%%%%%%%%%%%%%%%%%%%%%%%%%%%%%%%%%%%%%%%%%%%%%%%%%%%%%%%%%%%%%%%%%%%
\begin{frame}
\frametitle{Stability of Equilibria}

We use Jacobian Matrix,
\pause
\begin{equation}
\mathcal{J} =
\begin{bmatrix}
    \ -\R_0 I-\epsilon       & -\R_0 S \\
    \ \R_0 I       & \R_0 S-1 \\
\end{bmatrix} \,.
\end{equation}
\end{frame}
%%%%%%%%%%%%%%%%%%%%%%%%%%%%%%%%%%%%%%%%%%%%%%%%%%%%%%%%%%%%%%%
\begin{frame}
\frametitle{Stability of Equilibria}

Eigenvalues of the Jacobian are,
\begin{equation}
\begin{split}
\lambda_{1,2} = \frac{-(\epsilon K^2+2\epsilon K +4p\mathcal{R}_0)}{4K} \pm \\ \frac{\sqrt{(\epsilon K^2+2\epsilon K +4p\mathcal{R}_0)^2-4(2\epsilon K^3+8\epsilon Kp\mathcal{R}_0)}}{4K}
\end{split}
\end{equation}

Where $K = (\R_0 -1)+ \sqrt{(\R_0-1)^2+4\R_0 p}$\,,

We can conclude that $\Re(\lambda_{1,2})<0$
\end{frame}
%%%%%%%%%%%%%%%%%%%%%%%%%%%%%%%%%%%%%%%%%%%%%%%%%%%%%%%%%%%%%%%%%%
\begin{frame}
\frametitle{What else do we need?}
\pause
\begin{itemize}
\item We are unable to compare the results of this model to others for the purpose of observing any advantages.
\pause
\item One way of defining "have advantage" is to compare total mortality counts. It describes the total casualty.
\pause
\item If intentional infected cases have the same mortality rate as normally infected cases, then clearly it is going to be a disaster. Therefore, we need to have separate disease induce mortality rate for each of them.
\pause
\item We need to divide $I$ into two separate infective classes. $V$ for intentionally infected class, $I$ for normally infected class.
\end{itemize}
\end{frame}
%%%%%%%%%%%%%%%%%%%%%%%%%%%%%%%%%%%%%%%%%%%%%%%%%%%%%%%%%%%%%%%%%%
\begin{frame}
\frametitle{Model with disease induced mortality rate}

Therefore, our model becomes,
\begin{subequations}\label{eq:base_ODE}
\begin{align}
\dbyd{S}{\tau}&=\epsilon(1-p)- \R_0 S(V+I)-\epsilon S\,, \label{eq:S_by_tau}\\
\dbyd{V}{\tau}&=\R_0 SV+\epsilon p-V\,, \label{eq:V_by_tau}\\
\dbyd{I}{\tau}&=\R_0 SI-I\,, \label{eq:I_by_tau}\\
\dbyd{M}{\tau}&=\pmV(1-\epsilon) V+\pmI(1-\epsilon) I\,,\\
\dbyd{R}{\tau}&=(1-\pmV)(1-\epsilon) V+(1-\pmI)(1-\epsilon) I-\epsilon R\,,
\end{align}
\end{subequations}

Where $\pmV$ and $\pmI$ represent the mortality rate for intentionally infected and normally infected cases, respectively.
\end{frame}
%%%%%%%%%%%%%%%%%%%%%%%%%%%%%%%%%%%%%%%%%%%%%%%%%%%%%%%%%%%%%%%%%%
\begin{frame}
\frametitle{Equilibria}

If $p\neq 0$, the equilibrium is,
\begin{subequations}
\begin{align}
\hat{S}&= \frac{1}{\R_0}-\frac{2p}{(\R_0 -1)+ \sqrt{(\R_0-1)^2+4\R_0
         p}}\,, \label{eq:Shat}\\
\hat{V}&= \frac{\epsilon(\R_0 -1)+ \epsilon \sqrt{(\R_0-1)^2+4\R_0 p}}{2\R_0}\,, \label{eq:Vhat}\\
\hat{I}&=0\,. \label{eq:Ihat}
\end{align}
\end{subequations}

It is interesting that, at equilibrium, normally infected cases cease to exist. This may be helpful for eradication of disease.
\end{frame}
%%%%%%%%%%%%%%%%%%%%%%%%%%%%%%%%%%%%%%%%%%%%%%%%%%%%%%%%%%%%%%%%%%%
\begin{frame}
By using the same method as the previous model, we again showed that the equilibrium is stable. 
\end{frame}
%%%%%%%%%%%%%%%%%%%%%%%%%%%%%%%%%%%%%%%%%%%%%%%%%%%%%%%%%%%%%%%%%%%%%%
\begin{frame}
\frametitle{Effect of intentional infection on total mortality}
\begin{center}
Smallpox
\end{center}
\begin{table}[H]
\begin{center}
\caption{Model parameters and smallpox values.}
\label{tab:params}
\smallskip
\begin{tabular}{c|c|r}
{\bfseries Symbol} & {\bfseries Meaning} & {\bfseries Value} \\\hline
$\mu$ & Natural \emph{per capita} death rate & $\frac{1}{50\times365}$ per day \\
$\gamma$ & Recovery rate & $\frac{1}{22}$ per day \\
$\R_0$ & Basic reproductive number & 4.5\\
$\pmV$ & Intentionally infected cases death rate & 0.01\\
$\pmI$ & Normally infected cases death rate & 0.3
\end{tabular}
\end{center}
\end{table}
\end{frame}
%%%%%%%%%%%%%%%%%%%%%%%%%%%%%%%%%%%%%%%%%%%%%%%%%%%%%%%%%%%%%%%%%%%%%
\begin{frame}
\frametitle{Effect of intentional infection on total mortality}
\framesubtitle{Mortality rate at EE}
\begin{equation}
\left. \dbyd{M}{t}\right|_{\rm EE}=\frac{\pmV(1-\epsilon)\epsilon(\R_0 -1)+ \pmV(1-\epsilon)\epsilon \sqrt{(\R_0-1)^2+4\R_0 p}}{2\R_0}\,,
\end{equation}
\pause
\begin{itemize}
\item Mortality rate increases as $p$ increases.
\pause
\item In the long run, a larger proportion of intentional infection will lead to more casualties.
\end{itemize}
\end{frame}
%%%%%%%%%%%%%%%%%%%%%%%%%%%%%%%%%%%%%%%%%%%%%%%%%%%%%%%%%%%%%%%%%%%
\begin{frame}
\begin{figure}[H]
  \centering
  \includegraphics[width=0.7\textwidth]{Figures/Mortality_counts.png}
  \caption{$\dbyd{M}{\tau}$ at EE as a function of $p$.}
\label{fig:dMdt}
\end{figure}

Since the magnitude of $\dbyd{M}{\tau}$ is too small to be observed, we can hardly the the difference between the lines.
\end{frame}
%%%%%%%%%%%%%%%%%%%%%%%%%%%%%%%%%%%%%%%%%%%%%%%%%%%%%
\begin{frame}
To see the dynamics better, assume $\pmV=0.2$.
\begin{figure}[H]
  \centering
  \includegraphics[width=0.9\textwidth]{Figures/Rplot.pdf}
  \caption{$\dbyd{M}{\tau}$ at EE as a function of $p$.}
\label{fig:dMdt}
\end{figure}
\end{frame}
%%%%%%%%%%%%%%%%%%%%%%%%%%%%%%%%%%%%%%%%%%%%%%%%%%%%%
\begin{frame}
\frametitle{Initial state being at equilibrium}

In many historical cases, intentional infect is introduced when the population is at equilibrium, which is the equilibrium for $p=0$.
\pause
\begin{subequations}
\begin{align}
\hat{S} &= \frac{1}{\R_0} \,,\\
\hat{V} &= 0\,,\\
\hat{I} &= \epsilon(1-\frac{1}{\R_0})
\end{align}
\end{subequations}
\end{frame}
%%%%%%%%%%%%%%%%%%%%%%%%%%%%%%%%%%%%%%%%%%%%%%%%%%%
\begin{frame}
\frametitle{Initial state being at equilibrium}
We are interested in the time it takes to reach the new EE. We need to define a threshold for reaching equilibrium.
\pause

Since the new equilibrium has $\hat{I}=0$, we define reaching equilibrium $I\leq 1\times 10^{-6}$ (one in a million).
\end{frame}
%%%%%%%%%%%%%%%%%%%%%%%%%%%%%%%%%%%%%%%%%%%%%%%%%%%%%%
\begin{frame}
\begin{figure}[h]
  \centering
  \includegraphics[width=0.9\textwidth]{Figures/I_less_than_0_000001.pdf}
  \caption{Determination of time taken to reach equilibrium}
\end{figure}
\end{frame}
%%%%%%%%%%%%%%%%%%%%%%%%%%%%%%%%%%%%%%%%%%%%%%%%%%%%%5
\begin{frame}
\begin{figure}[H]
  \centering
  \includegraphics[width=0.9\textwidth]{Figures/dMdt.pdf}
  \caption{An illustration of intentional infection have advantages over non-intentional infection}
\end{figure}
\end{frame}
%%%%%%%%%%%%%%%%%%%%%%%%%%%%%%%%%%%%%%%%%%%%%%%%%%%%
\begin{frame}

To summarize the figures in the previous two slides,
\begin{table}[H]
\begin{center}
\caption{Time required to reach equilibrium and have advantages over non-intentional infection}
\label{tab:times}
\smallskip
\begin{tabular}{c|c|r}
{\bfseries $p$} & {\bfseries Time to EE} & {\bfseries Time to have advantages} \\\hline
0.1 & 4.27 yrs & 5.20 yrs \\
0.2 & 3.03 yrs & 8.81 yrs \\
0.4 & 2.27 yrs & 17.45 yrs \\
0.6 & 1.94 yrs & 28.37 yrs \\
0.8 & 1.74 yrs & 42.43 yrs \\
1.0 & 1.60 yrs & 61.46 yrs
\end{tabular}
\end{center}
\end{table}
\end{frame}
%%%%%%%%%%%%%%%%%%%%%%%%%%%%%%%%%%%%%%%%%%%%%%%%%%%%5
\begin{frame}
\begin{figure}[H]
  \centering
  \includegraphics[width=0.7\textwidth]{Figures/time_to_advantage_plot.pdf}
  \caption{Time to advantage, as a function of $p$}
\end{figure}
It seems like with a lower proportion of intentional infection, we can gain advantages relative faster.
\end{frame}
%%%%%%%%%%%%%%%%%%%%%%%%%%%%%%%%%%%%%%%%%%%%%%%%%%%%%
\begin{frame}
\frametitle{Possibility of disease eradication}

We know that $\hat{I}=0$ at EE. We could consider the normally infected cases already been eradicated.

\pause
How about intentionally infected cases? Can they burn out?
\end{frame}
%%%%%%%%%%%%%%%%%%%%%%%%%%%%%%%%%%%%%%%%%%%%%%%%%%%%%%
\begin{frame}
\frametitle{Possibility of disease eradication}

Vaccination threshold (Herd immunity): $p_c=1-\frac{1}{\R_0}$

\pause
\begin{center}
$S=\frac{1}{\R_0}$
\end{center}

\pause
If $S$ stay below this threshold until $V$ goes extinct, then we can achieve complete eradication of this disease.
\end{frame}
%%%%%%%%%%%%%%%%%%%%%%%%%%%%%%%%%%%%%%%%%%%%%%%%%%%%%
\begin{frame}
\frametitle{Possibility of disease eradication}

For example, if our initial intentional infection has a proportion $p=1$, then
\begin{figure}[H]
  \centering
  \includegraphics[width=0.75\textwidth]{Figures/Increase_of_S.pdf}
  \caption{For more than 10 years after we stop intentional infection, $S<\frac{1}{\R_0}$}
\label{figure:S_after_stop}
\end{figure}
\end{frame}
%%%%%%%%%%%%%%%%%%%%%%%%%%%%%%%%%%%%%%%%%%%%%%%%%%%%
\begin{frame}
\frametitle{Possibility of disease eradication}
\begin{figure}[H]
  \centering
  \includegraphics[width=0.8\textwidth]{Figures/V_after_stop.pdf}
  \caption{It takes less than 1 year for $V$ to fall below $1\times10^{-6}$}
\end{figure}
\end{frame}
%%%%%%%%%%%%%%%%%%%%%%%%%%%%%%%%%%%%%%%%%%%%%%%%%%%%%
\begin{frame}
\frametitle{Future expectation}
\begin{itemize}
\item Model extension
\pause
\begin{itemize}
\item Different transmission rate for intentionally or normally infected cases
\pause
\item Different recovery rate
\end{itemize}
\pause
\item Combination of strategies
\pause
\item Comparison to traditional vaccination
\pause
\item Other strategies for intentional infection
\end{itemize}
\end{frame}
%%%%%%%%%%%%%%%%%%%%%%%%%%%%%%%%%%%%%%%%%%%%%%%%%%%%%
\begin{frame}
\frametitle{Other strategies for intentional infection}

Another common strategy would be to intentionally infect susceptible individuals, with a certain rate.
\pause
\begin{equation}
\begin{split}
\dbyd{S}{t}&=\mu- \beta SI-rS-\mu S\,, \\
\dbyd{I}{t}&=rS+\beta SI-\gamma I -\mu I\,,\\
\dbyd{R}{t}&=\gamma I-\mu R\,,
\end{split}
\end{equation}
\end{frame}
%%%%%%%%%%%%%%%%%%%%%%%%%%%%%%%%%%%%%%%%%%%%%%%%%%%%%
\begin{frame}
\frametitle{Challenges to this method}
\begin{itemize}
\item Identify susceptible individuals
\item Compare with infecting newborn individuals, this method is largely relied on vaccination pattern (If we consider them using the same pattern),
\end{itemize}
\end{frame}
\end{document}

